\documentclass[11pt]{article}
\usepackage[utf8]{inputenc}
\usepackage[T1]{fontenc}
\usepackage{graphicx}
\usepackage{grffile}
\usepackage{longtable}
\usepackage{wrapfig}
\usepackage{rotating}
\usepackage[normalem]{ulem}
\usepackage{amsmath}
\usepackage{textcomp}
\usepackage{amssymb}
\usepackage{capt-of}
\usepackage{hyperref}

\usepackage{appendix}

\usepackage{titling}

\setlength{\droptitle}{-5em}

\author{Paloma Pedregal Helft}
\date{}
\title{Quickcheck: Property testing in Haskell}

\bibliographystyle{plain}
\begin{document}

\maketitle

\vspace{150px}

\tableofcontents

\clearpage

\section{Introduction}

This document is the report for the final proyect of the course \textit{Analysis of Concurrent Systems}.
The proyect's objective was to test a program using \textbf{Quickcheck}, a library for property testing.\\

The process started by writing a Haskell program. In this case, the program is a model of computer microarchitecture for studying different aspects of side-channel cache attacks.

After writing the program, the second step was listing the properties of the program, and the third implementing them in Quickcheck and testing the program.

This document consists of a first section explaining the program, a second with the properties and implementation of the tests, and finally a section with the results of the tests and a conclusion.

\clearpage

\section{A cache model in Haskell}

The program on which the tests were performed is a model of some parts of the microarchitecture of computers: Caches, different replacement policies, translation lookaside buffer (tlb). The model's objective is to study and reason about aspects of side-channel cache attacks.

\subsection{A primer on caches}

\clearpage

\section{Property testing}

\clearpage

\section{Results and conclusion}

\clearpage

\addcontentsline{toc}{section}{References}
\bibliography{bibliographyQuickcheck}

\clearpage


\end{document}
